% Template for Affective Computing and Intelligent Interaction (ACII)
%
% Modified 2022-03-15 : Desmond Ong (desmond.c.ong@gmail.com)     
%.      -- update for ACII2022, added section for Ethical Impact Statement
%

\documentclass[conference]{IEEEtran}
\IEEEoverridecommandlockouts
% The preceding line is only needed to identify funding in the first footnote. If that is unneeded, please comment it out.
\usepackage{cite} \usepackage{amsmath,amssymb,amsfonts}
\usepackage{algorithmic} \usepackage{graphicx} \usepackage{textcomp}
\usepackage{xcolor} \usepackage{hyperref} \usepackage[normalem]{ulem}

\def\BibTeX{{\rm B\kern-.05em{\sc i\kern-.025em b}\kern-.08em
    T\kern-.1667em\lower.7ex\hbox{E}\kern-.125emX}}
\begin{document}

\title{Emotion Twenty Questions Dialog System for Lexical Emotional
  Intelligence \thanks{
    %% We thank the following for support and
    %% funding: University of St. Thomas Center for Applied Artificial
    %% Intelligence and University of St. Thomas Center for Faculty
    %% Development Graduate Research Team Grant.
    We thank the following for support and funding {\it redacted for
      anonymity}.  }  }

\author{
\IEEEauthorblockN{1\textsuperscript{st} Anonymous Surname1}
\IEEEauthorblockA{\textit{dept. name of organization (of Aff.)} \\
\textit{name of organization (of Aff.)}\\
City, Country \\
email address or ORCID}
\and
\IEEEauthorblockN{2\textsuperscript{nd} Anonymous Surname2}
\IEEEauthorblockA{\textit{dept. name of organization (of Aff.)} \\
\textit{name of organization (of Aff.)}\\
City, Country \\
email address or ORCID}
\and
\IEEEauthorblockN{3\textsuperscript{rd} Anonymous Surname3}
\IEEEauthorblockA{\textit{dept. name of organization (of Aff.)} \\
\textit{name of organization (of Aff.)}\\
City, Country \\
email address or ORCID}


%% \IEEEauthorblockN{Abe Kazemzadeh}
%% \IEEEauthorblockA{\textit{Graduate Programs in Software} \\
%% \textit{University of St. Thomas}\\
%% St. Paul, MN, U.S.A. \\
%% %abe.kazemzadeh@stthomas.edu,
%% orcid:0000-0002-1851-294X}
%% \and
%% \IEEEauthorblockN{ Ola Sanusi}
%% \IEEEauthorblockA{\textit{Graduate Programs in Software} \\
%% \textit{University of St. Thomas}\\
%% St. Paul, MN, U.S.A. \\
%% ola.sanusi@stthomas.edu}
%% \and
%% \IEEEauthorblockN{Huihui (Summer) Nie}
%% \IEEEauthorblockA{\textit{Graduate Programs in Software} \\
%% \textit{University of St. Thomas}\\
%% St. Paul, MN, U.S.A. \\
%%}

%% \and
%% \IEEEauthorblockN{4\textsuperscript{th} Given Name Surname}
%% \IEEEauthorblockA{\textit{dept. name of organization (of Aff.)} \\
%% \textit{name of organization (of Aff.)}\\
%% City, Country \\
%% email address or ORCID}
%% \and
%% \IEEEauthorblockN{5\textsuperscript{th} Given Name Surname}
%% \IEEEauthorblockA{\textit{dept. name of organization (of Aff.)} \\
%% \textit{name of organization (of Aff.)}\\
%% City, Country \\
%% email address or ORCID}
%% \and
%% \IEEEauthorblockN{6\textsuperscript{th} Given Name Surname}
%% \IEEEauthorblockA{\textit{dept. name of organization (of Aff.)} \\
%% \textit{name of organization (of Aff.)}\\
%% City, Country \\
%% email address or ORCID}
}

\maketitle

\begin{abstract}
  %A brief background introduction and description of the system:
  
  This paper presents a web-based demonstration of Emotion Twenty
  Questions (EMO20Q), a dialog game whose purpose is to study how
  people describe emotions.  EMO20Q can also be used to develop
  artificially intelligent
  % or artificial intelligent 
  dialog agents that can play the game.  In
  previous work, an EMO20Q agent used a sequential Bayesian machine
  learning model and could play the role of asking questions.  Newer
  transformer-based neural machine learning models have made it
  possible to develop an agent for the question-answering role.

  %a summary of the technical contributions of the system:

  This demo paper describes the recent developments in the
  question-answering role of the EMO20Q game, which requires the agent
  to respond to more open-ended inputs.  Furthermore, we also describe
  the design of the system, including the web-based front-end, agent
  architecture and programming, and updates to earlier software used.

  %results of a formal evaluation or experiment to be conducted during
  %the demo, if applicable:

  The demo system will be available to collect pilot data during
  the ACII conference and this data will be used to inform future
  experiments and system design.

  % Our early results indicate that ...

  %a url to a video recording or screenshots of how the system works:
    %what would be a good url for emo20q?  emo20q.org? emo20q.ai?

  An example of the demo system can be seen in
  Fig.~\ref{fig:emo20q-flask-websocket} and the live system can be
  accessed at \href{https://emo20q.org}{emo20q.org}.
\end{abstract}

\begin{IEEEkeywords}
emotions, natural language processing, dialog systems,
question-answering, EMO20Q, lexicon
\end{IEEEkeywords}

\begin{figure}[h]
\centering
\includegraphics[width=0.5\textwidth]{emo20q-flask-websocket}
\caption{Example of the web-based front end to the EMO20Q dialog
  system.}
\label{fig:emo20q-flask-websocket}
\end{figure}


\section{Introduction}

This demo system aims to use a dialog agent to collect question-answer
data about human emotions.  We use the emotion twenty questions game
(EMO20Q) as the experimental setting.  In EMO20Q, one player picks an
emotion word and the other player tries to guess it in twenty or fewer
turns. The player that picks the emotion word plays the {\it
  question-answering} role, because they answer questions about the
emotion word that they picked.  The player who tries to guess the
emotion word plays the {\it question-asking} role because they ask
questions in order to identify the emotion word.

Both player roles for EMO20Q could be played by humans or computers.
Past research collected data of humans playing both roles and then
used this data to train dialog agents to play the question-asking role
\cite{Kazemzadeh2012}.  This past work used a sequential Bayesian
probabilistic model for the question-asking role in the game
\cite{Kazemzadeh2012}: the input was a sequence of question-answer
pairs. The question set comes from human-human dialogs and the
particular question that the agent asks is chosen based on the dialog
history. The answers come from the user's corresponding responses.
Posterior probabilities of the emotion words are updated based on
these sequential question-answer inputs.
% move based on Summer's suggestion?
%% In our previous work \cite{Kazemzadeh2016} work, we considered the
%% questioner role to be easier because the set of questions was
%% predetermined based on the human-human dialogs, so the only open-ended
%% input is the answers to yes-no questions.  These are not restricted to
%% yes or no, but it is trivial to bucket these responses into
%% yes/no/other categories.

The current work aims to extend the approach to a dialog agent that
can play the question-answering role. In this case, the agent picks
the emotion word from words seen in previously collected data (167
words) and answers questions posed by the user. In this case, the
agent's inputs are very open-ended: any possible yes/no questions
about emotions.  Many questions, especially earlier in the game, have
been seen before (e.g., ``is it a positive emotion?'', ``is it a
negative emotion?'', ``is it an emotion that is directed at another
person?'', ``is it an emotion that lasts a long time?''), but in
principle, any questions could be asked.  In comparison, the inputs to
the question-asking role are more limited.  The question-asking agent
asks questions from the set of questions seen in human-human data and
its inputs are answers to these yes-no questions. The answers are not
restricted to yes or no, but it is fairly trivial to bucket these
responses into yes/no/other categories.

To deal with the open-ended question-answering task, we aim to
leverage neural large language models (LLM), which encode linguistic
knowledge from large pre-training datasets into neural networks,
which can then be fine-tuned into task specific models with relatively
smaller datasets.  % currently using a place holder/basic model and will


The goal for EMO20Q dialog system is to use dialog agents to test
hypotheses about how well automated systems can understand language
about emotions.

%% The current system still involves an automated agent
%% playing with a human user.  However, with dialog agents that can play
%% both both roles, two automated agents could play EMO20Q with each
%% other.


%% \section{Stash}
%% \begin{itemize}
%% \item \href{https://acii-conf.net/2022/calls/call-for-demos-2/}{call for demos}
%% \item \href{https://acii-conf.net/2022/authors/submission-guidelines/}{submission guidelines}
%% \item submit using
%%   \href{https://easychair.org/conferences/?conf=aciioo2022}{EasyChair}
%% \end{itemize}

\section{Web Front End}

To simply display the front end of the demo system, we used Flask, a
lightweight Python web framework.  Using the simple request-response
interaction of a basic web server would have been an option, but it
would require a page reload for every dialog turn and could result in
slow response times, especially in cases where prediction requires
beam search (the question-answering role).  To prevent page reloads
and response delays, we used WebSockets via the Socket.io library,
which enables bidirectional communication between the browser and
web server.  Dialog events, i.e. turns from the user (via the browser)
or agent (via the server), trigger updates to a typical speech bubble
dialog display, as seen in Fig.~\ref{fig:emo20q-flask-websocket}.

\section{Dialog Agent}

The dialog agent was trained using a method inspired by wizard of Oz
approaches \cite{Fraser1991} and games with a purpose
\cite{Ahn2004}. First, non-expert human players played both roles of
EMO20Q, which provided an initial set of training data. Then, we
created automated agent for the question-asking role by finding the
conditional probabilities of (question, answer) pairs given emotion
words and using these to create a sequential Bayesian probabilistic
model \cite{Kazemzadeh2012}, which provided a further source of
training data.  The current work has focused on the question-answering
role.  This current work has benefited from recent trends in
transfer-learned deep neural language models.  In particular
%, we used BERT, an encoder only system, and
%% T5 \cite{Raffel2020}, an encoder-decoder pretrained language model for
%% this demo system, though a more systematic examination of other models
%% is planned for future work.
this demo uses BERT \cite{Devlin2019} to classify (emotion, question)
pairs into ``yes'', ``no'', and ``maybe'' categories. Although our
system achieves reasonable performance of about 72\% accuracy on
question-answering, we plan that the demo system will be an
experimental tool to more fully evaluate this and other models.

The programming of the agent is based on a generalized pushdown
automaton (GPDA) \cite{Allauzen2012}.  The abilities of a general
finite state automata are sufficient for most of the dialog behavior,
but the pushdown stack is used to maintain contextual information that
represents short-term/working memory in the question-asking role. The
transition graph of the automaton is implemented with NetworkX Python
library.


\section{Discussion and Future Work}

This paper presented an overview of a tool that we will use to test
the abilities of automated systems to talk about emotions with human
users.  In this demo, we use BERT, an encoder-only model set up as a
classification model to provide yes/no/maybe answers.  This
classification approach fits with the existing approached used by the
question-asking agent. Other transformer models that include decoders
may allow for more fluent output and may be able to encompass the
functionality currently covered by the dialog graph,  so we anticipate
that this will be an area of future testing.

Although the agent in our demo appears to the user as a single agent
playing both roles, the design of the functionality of this single
agent is currently implemented as two separate agents with different
machine learning models built on the same data. One area of future
research is to design a more unified approach where machine learning
models are shared between the two EMO20Q roles.  This approach could
enable adversarial training scenarios where two automated agents play
each other.


%% Django vs Flask. Flask is simpler to use but lacks the standardized
%% user management functions (authentication/login, emails), model-view
%% controller-framework, database ORM, etc., that comes included in
%% Django.  While the data generated from the dialogs is json, other
%% features of a RDBMS system would be useful in cases where structured
%% data is maintained (more digested data like emotion vocabularies, user
%% information for demographic information and querying user history).

%% If we use the decoder generation as output, what should the generation
%% params be?  greedy, beam, etc

Another issue is the ordering of EMO20Q roles.  Currently, the demo
system makes the user play the question-answering role first, then the
question-asking role. This order may induce experimental effects.  For
example, playing as question-answerer first may prime the human user
to reuse the agent's questions.

%% We intend to use this demo system to collect data and use it to test
%% models and hypotheses primarily for the question-asking role.  Currently,
%% the system implements a basic question-answering model by fine tuning the
%% pretrained BERT model.  Using the demo system, we hope to test and improve this model as well as 

In conclusion, we demonstrate an automated system that plays EMO20Q
using a web-based chat interface and a mix of older probabilistic
models for the question-asking role and a fine-tuned BERT model for
the question-answering role.


%Also, the game
%setting may be a possible application for adversarial training.


\section{Ethics Statement}

This research has been approved by the
%University of St. Thomas
{\it redacted for anonymity}
institutional review board (IRB).  Data is not collected with
personally identifiable information (PII) and the collected with be
examined for PII before public dissemination.



\bibliographystyle{ieeetr}
\bibliography{../../../Dropbox/org/reference/AbesBigBibliography}
\end{document}
